\documentclass[12pt]{article}

\usepackage{amsmath,amssymb,latexsym,enumitem,qtree,comment,color,ifthen}
\usepackage{algorithmicx,algorithm,algpseudocode}

\setlength{\oddsidemargin}{-0.5in}
\setlength{\evensidemargin}{-0.5in}
\setlength{\textwidth}{7.5in}
\setlength{\topmargin}{0in}
\setlength{\textheight}{8.5in}

\newcommand{\up}[1]{\vspace*{-#1mm}}
\renewcommand{\arraystretch}{1.8}
\newcommand{\qed}{\hfill$\square$}
\newcommand{\assignNum}{6}
\newcommand{\numPoints}{40}
\newcommand{\dueDate}{Monday 10/15/2018}
\newcommand{\pbStatement}[2]{
  \colorbox{yellow}{\parbox[t]{0.93\textwidth}
        {\bf (#1 \ifthenelse{#1 > 1}{points}{point})
          #2}}}
\newcommand{\morePbStatement}[1]{
  \colorbox{yellow}{\parbox[t]{0.93\textwidth}
        {\bf #1}}}
\newcommand{\fillIn}[1]{\fbox{\parbox[t]{0.93\textwidth}{\it #1}}}
\newcommand{\header}{
\noindent{\textit{CS 321 - Algorithms - Fall 2018\hfill \yourName}}
\begin{center}
  \textbf{\Large Assignment \assignNum}\\\bigskip
  \colorbox{red}{\Large\bf
     \textcolor{white}{Due BEFORE 8:00AM on  \dueDate}}\\\bigskip

  On time /  20\% off / no credit\\\bigskip

  \textbf{Total points: \numPoints}
  \bigskip

  \hrule\medskip You are allowed to work with a partner on this
  assignment. If you decide to form a pair, make sure to include
  both names above, but submit only one zip file to D2L.
  \medskip \hrule \end{center}

\medskip
This assignment will develop your algorithmic design skills. More
specifically, it will  make you more comfortable with the divide-and-conquer
paradigm.  You must write up your solutions to this assignment IN THIS
FILE using \LaTeX\ by filling in all of the boxes below. If your
submitted \texttt{.tex} file does not compile, then you will receive 0
points. You should NOT add any \LaTeX\ packages to your \texttt{.tex}
file.\newline

Since this assignment includes a programming component, you must also
complete and submit the \texttt{SLIT.java} file.




\medskip
\textbf{Submission procedure:}
\begin{enumerate}[itemsep=-2mm]

\item Complete this file, called \texttt{a\assignNum .tex}, with your
  full name(s) and answers typed up below.

\item Compile this file to produce a file called \texttt{a\assignNum .pdf}. Make
sure that this file compiles properly and that its contents and appearance meet
the requirements described in this handout.

\item Create a directory called \texttt{a\assignNum} and copy exactly
  three files into this directory, namely:\up{4}

  \begin{itemize}[itemsep=-1mm]
  \item \texttt{a\assignNum .tex} (this file with all of your answers added)
  \item \texttt{a\assignNum .pdf} (the compiled version of the file above)
  \item \texttt{SLIT.java} (your completed Java code file)
  \end{itemize}

\item Zip up this directory to yield a file called \texttt{a\assignNum .zip}

\item Submit this zip file to the D2L dropbox for A\assignNum\  before the
deadline above.

\item BEFORE the beginning of class on the due date above, submit a
single-sided, hard copy of your:
  \begin{itemize}
    \item \texttt{a\assignNum .pdf} file
    \item completed \texttt{SLIT.java} file, making sure that your name(s) are
      in the top documentation block and that you stated whether or not each
      algorithm works as expected. For FULL credit, your hard copy of the code
      MUST be properly indented, contain no lines that wrap around (i.e., with
      a length greater than the page width), and be fully legible
      and easily understandable.
  \end{itemize}

\end{enumerate}
}% end of header command

%************************************
% No need to modify anything above this line
% but DO fill this in!

\newcommand{\yourName}{\fbox{Justin Espiritu Robert Freiberg}}

%************************************g

\begin{document}
\header
\newpage

\textbf{Problem statement}\medskip

In this assignment, you will implement two solutions to the following
problem:\medskip

\noindent\fbox{\centerline{\bf SLIT problem}}
\begin{description}[align=left]
  \item [\bf Input:] An $n\times n$ matrix $m$ whose elements come
    from the set \{`{\tt A}', `{\tt C}', `{\tt G}', `{\tt T}'\}
  \item [\bf Output:] The size $s$ of the largest imbalance in Thymine
    within any slit window in $m$, where:\up{2}
    \begin{itemize}
    \item a slit window is a rectangular sub-matrix of $m$ with height 2
      and any positive width, and
    \item $s$ is computed as the total number of `{\tt T}' elements
      minus the total number of `{\tt A}', `{\tt C}', and `{\tt G}'
      elements in the window.
    \end{itemize}
\end{description}
\hrule\bigskip

The first solution you will implement is given to you in pseudocode
format, as follows:

\begin{algorithm}[H]
    \caption{Brute-force SLIT algorithm}
    \label{euclid}
    \begin{algorithmic}[1] % number = starting value for line numbers

      \Function{count}{$m,r,c,w$}

      \Comment{\begin{minipage}{0.93\linewidth}This is a helper
          function that returns the total number of non-T bases
          subtracted from the total number of T bases within the
          sub-matrix of $m$ whose top-left corner is at $r\times c$,
          whose width is equal to $w$ columns, and whose height is
          equal to two rows\end{minipage}}
      \EndFunction
      \Function{BF\_SLIT}{$m,n$} \Comment{m: 2D matrix}

      \Comment{n: \# of top rows and left columns of $m$ that are part of this
      problem instance}

      \State $slit \gets 0$
      \For{each $row \in[0..(n-2)]$}
      \Comment{$row$ is the index of the row in the top-left corner
        of the slit}
          \For{each $col \in [0..(n-1)]$}
          \Comment{$col$ is the index of the column in the top-left
            corner of the slit}
                 \For{each $w \in  [1..(n-col)]$}
                  \Comment{$w$ is the number of columns in the slit}
                        \State $slit \gets max(slit,count(m,row,col,w))$
                 \EndFor
          \EndFor
      \EndFor
      \State \textbf{return} $slit$
      \EndFunction
    \end{algorithmic}
\end{algorithm}


You will have to design another algorithm that has a better
asymptotic running time than the brute-force algorithm given above.
Here are the detailed requirements for this assignment.

\begin{enumerate}

  %%%%%%%%%%%%%%%%%%%%%%%%%%%%%%%%%%%%%%%%%%%%%%%%%%%%%%%%%%%%%%%%%%%%%%%%%%%
  \item                            % Question 1
  %%%%%%%%%%%%%%%%%%%%%%%%%%%%%%%%%%%%%%%%%%%%%%%%%%%%%%%%%%%%%%%%%%%%%%%%%%%
     {\bf \color{red} (5 points) What is the big-theta bound on the worst-case
       running time of the brute-force algorithm?}

  %%%%%%%%%%%%%%%%%%%%%%%%%%%%%%%%%%%%%%%%%%%%%%%%%%%%%%%%%%%%%%%%%%%%%%%%%%%
     Answer: $\Theta(N^4)$
  %%%%%%%%%%%%%%%%%%%%%%%%%%%%%%%%%%%%%%%%%%%%%%%%%%%%%%%%%%%%%%%%%%%%%%%%%%%

  {\bf \color{red} Prove your answer by computing the exact worst-case
  running time.}

  %%%%%%%%%%%%%%%%%%%%%%%%%%%%%%%%%%%%%%%%%%%%%%%%%%%%%%%%%%%%%%%%%%%%%%%%%%%
    \begin{math}
        \begin{array}{llll}
            T(N) & = & \sum_{r=0}^{n-2} \sum_{c=0}^{n-`1} \sum_{w=1}^{n-c} \sum_{i=1}^2 \sum_{j=1}^w 1 & \text{taken from the SLIT algorithm} \\
            & = & \sum_{r=0}^{n-2} \sum_{c=0}^{n-`1} \sum_{w=1}^{n-c} \sum_{i=1}^2 \frac{w(w+1)}{2} & \text{geometric series} \\
            & = & \sum_{r=0}^{n-2} \sum_{c=0}^{n-`1} \sum_{w=1}^{n-c} 2 \frac{w(w+1)}{2} & \text{simplify via property of summation} \\
            & = & \sum_{r=0}^{n-2} \sum_{c=0}^{n-`1} \sum_{w=1}^{n-c} \sum_{i=1}^2 w(w+1) & \text{simplify}\\
            & = & \sum_{r=0}^{n-2} \sum_{c=0}^{n-`1} \sum_{w=1}^{n-c} \sum_{i=1}^2 w^2 + w & \text{simplify} \\
            & = & \sum_{r=0}^{n-2} \sum_{c=0}^{n-`1} (\sum_{w=1}^{n-c} w^2 + \sum_{w=1}^{n-c} w) & \text{distribute summation} \\
            & = & \sum_{r=0}^{n-2} \sum_{c=0}^{n-`1} (\frac{(n-c)((n-c) +1)(2(n-c)+1)}{6} + \frac{(n-c)((n-c)+1)}{2}) & \text{geometric series} \\
            & = & \sum_{r=0}^{n-2} \sum_{c=0}^{n-1} \frac{2(n-c)^3+6(n-c)^2+6(n-c)+3(n-c)^2+3(n-c)}{6} & \text{simplify}\\
            & = & \sum_{r=0}^{n-2} \sum_{c=0}^{n-1} \frac{2n^3-c^3-3cn^2+3c^2n+9n^2-18cn+9c^2+9n-9c}{6} & \text{distribute and simplify}\\
            & = & \sum_{r=0}^{n-2} \sum_{c=0}^{n-`1} \frac{n^3}{3} - \frac{c^3}{6} - \frac{cn^2}{2} + \frac{c^2n}{2} + \frac{3n^2}{2} + \frac{3c^2}{2} + \frac{3n}{2} - \frac{3c}{2} & \text{distribute the 6}\\
            & = & \sum_{r=0}^{n-2} (\frac{n^3}{3} - \frac{1}{6}\sum_{c=0}^{n-`1} c^3 - \frac{n^2}{2}\sum_{c=0}^{n-`1} c \\
            && + \frac{n}{2} \sum_{c=0}^{n-`1} c^2 - 3n\sum_{c=0}^{n-`1} c + \frac{3}{2}\sum_{c=0}^{n-`1} c^2 \\
            && + \frac{3n}{2} - \frac{3}{2}\sum_{c=0}^{n-`1} c) & \text{property of summations} \\
            & = & \sum_{r=0}^{n-2} (\frac{n^3}{3} - \frac{1}{6} \frac{(n-1)^2n^2}{24} - \frac{n^2}{2}(\frac{(n-1)n}{2})\\
            && + \frac{n}{2} (\frac{n(n-1)(2(n-1)+1)}{6}) - 3n (\frac{n(n-1)}{2}) + \frac{3}{2}(\frac{n(n-1)(2(n-1)+1)}{6}) \\
            && + \frac{3n}{2} - \frac{3}{2}(\frac{n(n-1)}{2})) & \text{simplify, property of summations} \\
            &=& \sum_{r=0}^{n-2} \frac{n^3}{3} - \frac{n^4 - 2n^3 - n^2}{24} - \frac{n^4 - n^3}{4} \\
            && + \frac{2n^4 - 3n^3 + n^2}{12} - \frac{3n^3 - 3n^2}{2} + \frac{6n^3 - 9n^2 - 3}{12} + \frac{3n}{2} - \frac{3n^2 - 3n}{4} & \text{simplify} \\
            &=& \sum_{r=0}^{n-2}\frac{-3n^4 - 14n^3 + 3n^2 + 54n - 6}{24} & \text{simplify}\\
            &=&\frac{-3n^4 - 14n^3 + 3n^2 + 54n - 6}{24} & \text{simplify, property of summations}\\
            &=& \frac{-n^4}{8} - \frac{-7n^3}{12} + \frac{n^2}{8} + \frac{9n}{4} - \frac{1}{4} & \text{simplify}\\
            \therefore & = & \Theta{n^4} & \text{simplify, property of Theta}
        \end{array}
    \end{math}
  %%%%%%%%%%%%%%%%%%%%%%%%%%%%%%%%%%%%%%%%%%%%%%%%%%%%%%%%%%%%%%%%%%%%%%%%%%%

  %%%%%%%%%%%%%%%%%%%%%%%%%%%%%%%%%%%%%%%%%%%%%%%%%%%%%%%%%%%%%%%%%%%%%%%%%%%
  \item                            % Question 2
  %%%%%%%%%%%%%%%%%%%%%%%%%%%%%%%%%%%%%%%%%%%%%%%%%%%%%%%%%%%%%%%%%%%%%%%%%%%
   {\bf \color{red} (6 points) Implement the brute-force algorithm, that is, the
   two methods called {\tt count} and {\tt algorithm1} in the file {\tt
   SLIT.java}. For this question, you may NOT modify any other part of the code
   handout besides the BODY OF THESE TWO METHODS.}\medskip

\fillIn{Your answer to this problem will be fully contained in the
  {\tt SLIT.java} file that you submit}

  %%%%%%%%%%%%%%%%%%%%%%%%%%%%%%%%%%%%%%%%%%%%%%%%%%%%%%%%%%%%%%%%%%%%%%%%%%%
  \item                            % Question 3
  %%%%%%%%%%%%%%%%%%%%%%%%%%%%%%%%%%%%%%%%%%%%%%%%%%%%%%%%%%%%%%%%%%%%%%%%%%%

    {\bf \color{red} (8 points) Design an algorithm whose worst-case running time is
    asymptotically better than your answer to the previous question. Your
    algorithm must use the divide-and-conquer approach. You must describe in
    precise and concise English prose the main idea(s) behind your algorithm by
    filling in the following four descriptions.}

   {\bf \color{red} a. Divide step}

   %%%%%%%%%%%%%%%%%%%%%%%%%%%%%%%%%%%%%%%%%%%%%%%%%%%%%%%%%%%%%%%%%%%%%%%%%%%
      \fillIn{
       We are going through the n by n matrix and we are going through each row. Each row starting at row 0 until n - 2. Our divide step does not start until we call maxTRec. MaxTRec needs the matrix m, the start of the window l, the end of the window r and what row we are in called row.


      We start by looking at the window whose size is given by the arguments of maxTRec. The window starts at l and ends at r. We then split that window by 2. The first half of the windows starts at l and goes to (l+r)/2 which we call mid. The other array starts at mid+1 and then goes to r. We keep doing this recursively until we reach our base case of l=r. Once we reach our base we then call the helper method count which moves us to the conquer step. }
   %%%%%%%%%%%%%%%%%%%%%%%%%%%%%%%%%%%%%%%%%%%%%%%%%%%%%%%%%%%%%%%%%%%%%%%%%%%

   {\bf \color{red} b. Conquer step}

   %%%%%%%%%%%%%%%%%%%%%%%%%%%%%%%%%%%%%%%%%%%%%%%%%%%%%%%%%%%%%%%%%%%%%%%%%%%
      \fillIn{Calculate the count of the window.}
   %%%%%%%%%%%%%%%%%%%%%%%%%%%%%%%%%%%%%%%%%%%%%%%%%%%%%%%%%%%%%%%%%%%%%%%%%%%

   {\bf \color{red} c. Combine step}

   %%%%%%%%%%%%%%%%%%%%%%%%%%%%%%%%%%%%%%%%%%%%%%%%%%%%%%%%%%%%%%%%%%%%%%%%%%%
      \fillIn{At the combine step we have two counts the right w$_{1}$ count and the left w$_{2}$ count. We then calculate the max count of the entire window. We then take the max of w$_{1}$, w$_{2}$, and the max count of w$_{1} + $w$_{2}$.}
   %%%%%%%%%%%%%%%%%%%%%%%%%%%%%%%%%%%%%%%%%%%%%%%%%%%%%%%%%%%%%%%%%%%%%%%%%%%

   {\bf \color{red} d. Base case(s)}

   %%%%%%%%%%%%%%%%%%%%%%%%%%%%%%%%%%%%%%%%%%%%%%%%%%%%%%%%%%%%%%%%%%%%%%%%%%%
     \fillIn{We have two base cases. The base case for Algorithm2 is when there is matrix of n by n where n $<$ 2 we return 0. The other base case is of our helper function maxTRec. The base case is when l = r we then return the count of that window.

     }
   %%%%%%%%%%%%%%%%%%%%%%%%%%%%%%%%%%%%%%%%%%%%%%%%%%%%%%%%%%%%%%%%%%%%%%%%%%%

 %%%%%%%%%%%%%%%%%%%%%%%%%%%%%%%%%%%%%%%%%%%%%%%%%%%%%%%%%%%%%%%%%%%%%%%%%%%
 \item                            % Question 4
 %%%%%%%%%%%%%%%%%%%%%%%%%%%%%%%%%%%%%%%%%%%%%%%%%%%%%%%%%%%%%%%%%%%%%%%%%%%
    {\bf \color{red} (7 points) Fully specify your algorithm in pseudocode
    format using the \LaTeX\ packages included in this file (just like in
    earlier assignments). Each recursive call of your algorithm must use $O(1)$
    additional space.}

   %%%%%%%%%%%%%%%%%%%%%%%%%%%%%%%%%%%%%%%%%%%%%%%%%%%%%%%%%%%%%%%%%%%%%%%%%%%





    \begin{algorithm}[H]
    \caption{Our SLIT algorithm}
    \label{euclid}
    \begin{algorithmic}[1] % number = starting value for line numbers

      \Function{algorithm2}{$m,n$}
        \If{n$<$2} return 0
        \EndIf

        \State slit = 0
        \For{row = 0; row $<$= n-2; row++}
            \State slit = Math.max(slit, maxTRec(m, 0, n, row)
        \EndFor
      \EndFunction


      \Function{maxTRec}{$m,l,r,row$}

      \If{l == r} return count(m, row, l, 1)
      \EndIf
        \State mid = (l+r)/2
        \State maxL = maxTRec(m, l, mid, row)
        \State maxR = maxTRec(m, mid + 1, r, row)
        \State maxSumL = 0
        \State maxSumR = 0
        \State sum = 0

      \For{(i = mid; i $>$= l; $i--$)}
        \State  sum += count(m, row, i, 1)
        \If{sum $>$ maxSumL} maxSumL = sum
        \EndIf
      \EndFor
      \State sum = 0
          \For{j = mid + 1; j $<$ r; j++}
            \State  sum += count(m, row, j, 1)
            \If{sum $>$ maxSumR} maxSumR = sum
        \EndIf
      \EndFor
      \State \textbf{return} return Math.max(Math.max(maxL, maxR), maxSumL + maxSumR)
      \EndFunction
    \end{algorithmic}
\end{algorithm}

   %%%%%%%%%%%%%%%%%%%%%%%%%%%%%%%%%%%%%%%%%%%%%%%%%%%%%%%%%%%%%%%%%%%%%%%%%%%

 %%%%%%%%%%%%%%%%%%%%%%%%%%%%%%%%%%%%%%%%%%%%%%%%%%%%%%%%%%%%%%%%%%%%%%%%%%%
 \item                            % Question 5
 %%%%%%%%%%%%%%%%%%%%%%%%%%%%%%%%%%%%%%%%%%%%%%%%%%%%%%%%%%%%%%%%%%%%%%%%%%%
      {\bf \color{red} (2 points) What is the FULL DEFINITION of the recurrence
      that describes your algorithm?}

   %%%%%%%%%%%%%%%%%%%%%%%%%%%%%%%%%%%%%%%%%%%%%%%%%%%%%%%%%%%%%%%%%%%%%%%%%%%
     Answer:
     \begin{math}
     T(N) = \sum_{r=0}^{n-2}2T(\frac{nt2}{2}) + 2n
     \end{math}

   %%%%%%%%%%%%%%%%%%%%%%%%%%%%%%%%%%%%%%%%%%%%%%%%%%%%%%%%%%%%%%%%%%%%%%%%%%%

 %%%%%%%%%%%%%%%%%%%%%%%%%%%%%%%%%%%%%%%%%%%%%%%%%%%%%%%%%%%%%%%%%%%%%%%%%%%
 \item                            % Question 6
 %%%%%%%%%%%%%%%%%%%%%%%%%%%%%%%%%%%%%%%%%%%%%%%%%%%%%%%%%%%%%%%%%%%%%%%%%%%
      {\bf \color{red} (2 points) What is the solution to the recurrence you
      gave as an answer to Question 5 above?}

   %%%%%%%%%%%%%%%%%%%%%%%%%%%%%%%%%%%%%%%%%%%%%%%%%%%%%%%%%%%%%%%%%%%%%%%%%%%
     Answer: $\Theta$( \fbox{
     \begin{math}
     N^2\log{}{N}
     \end{math}})
   %%%%%%%%%%%%%%%%%%%%%%%%%%%%%%%%%%%%%%%%%%%%%%%%%%%%%%%%%%%%%%%%%%%%%%%%%%%

  %%%%%%%%%%%%%%%%%%%%%%%%%%%%%%%%%%%%%%%%%%%%%%%%%%%%%%%%%%%%%%%%%%%%%%%%%%%
  \item                            % Question 7
  %%%%%%%%%%%%%%%%%%%%%%%%%%%%%%%%%%%%%%%%%%%%%%%%%%%%%%%%%%%%%%%%%%%%%%%%%%%
    {\bf \color{red} (8 points) Implement your algorithm, that is, the body of the
      method called {\tt algorithm2} in the file {\tt SLIT.java} as
      well as any helper methods that you need (these, MUST be added
      just below the {\tt algorithm2} method). For this question, you
      may NOT modify any other part of the code handout. Recall that each
      recursive call of your algorithm must use $O(1)$ additional space.}

    \fillIn{Your answer to this problem will be fully contained in the
      {\tt SLIT.java} file that you submit}

  %%%%%%%%%%%%%%%%%%%%%%%%%%%%%%%%%%%%%%%%%%%%%%%%%%%%%%%%%%%%%%%%%%%%%%%%%%%
  \item                            % Question 7
  %%%%%%%%%%%%%%%%%%%%%%%%%%%%%%%%%%%%%%%%%%%%%%%%%%%%%%%%%%%%%%%%%%%%%%%%%%%
    {\bf \color{red} (2 points) Run the driver code provided in {\tt A6.java}
      without modifying it. You MUST use the seed value of 1 in the
      output that you include below. You may interrupt the execution
      of the driver once the brute-force algorithm takes more than
      five minutes to complete. Include the output of this run below.
      }

    \fillIn{Delete this box; then copy and paste the full output of your code
    below within a {\tt verbatim} environment. }

    %%%%%%%%%%%%%%%%%%%%%%%%%%%%%%%%%%%%%%%%%%%%%%%%%%%%%%%%%%%%%%%%%%%%%%%%%%%
\begin{verbatim}
Populating the matrix...
Done
seed = 1
maxN = 8192
    32,   0.003,  22,      22,  0.001
    64,   0.015,  36,      36,  0.001
   128,   0.188,  76,      76,  0.002
   256,   3.542, 124,     124,  0.010
   512,  70.699, 196,     196,  0.042
\end{verbatim}
    %%%%%%%%%%%%%%%%%%%%%%%%%%%%%%%%%%%%%%%%%%%%%%%%%%%%%%%%%%%%%%%%%%%%%%%%%%%

  \end{enumerate}

  \end{document}
